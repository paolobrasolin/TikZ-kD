%%%%%%%%%%%%%%%%%%%%%%%%%%%%%%%%%%%%%%%%%%%%%%%%%%%%%%%%%%%%%%%%%%%%%%%%%%%%%%%%
%                                                                              %
%  Copyright 2015 by Paolo Brasolin <paolo.brasolin@gmail.com>                 %
%                                                                              %
%  This file is part of kD.                                                    %
%                                                                              %
%  kD is free software: you can redistribute it and/or modify                  %
%  it under the terms of the GNU General Public License as published by        %
%  the Free Software Foundation, either version 3 of the License, or           %
%  (at your option) any later version.                                         %
%                                                                              %
%  kD is distributed in the hope that it will be useful,                       %
%  but WITHOUT ANY WARRANTY; without even the implied warranty of              %
%  MERCHANTABILITY or FITNESS FOR A PARTICULAR PURPOSE.  See the               %
%  GNU General Public License for more details.                                %
%                                                                              %
%  You should have received a copy of the GNU General Public License           %
%  along with kD.  If not, see <http://www.gnu.org/licenses/>.                 %
%                                                                              %
%%%%%%%%%%%%%%%%%%%%%%%%%%%%%%%%%%%%%%%%%%%%%%%%%%%%%%%%%%%%%%%%%%%%%%%%%%%%%%%%

\pgfqkeys{/cleanser/expand}{.is choice,
  none/.style={/cleanser/input/.style={/cleanser/process={##1}}},
  once/.style={/cleanser/input/.style={/cleanser/process/.expand once={##1}}},
  full/.style={/cleanser/input/.style={/cleanser/process/.expanded={##1}}}}

\pgfqkeys{/cleanser}{
  expand=none,
  output/.initial={},
  process/.code={
    \kDSanitize#1\into tmp\GO
    \message{[Cleanser: \Meaning-> \tmp]}
    \pgfkeysalso{/cleanser/output/.expand once=\tmp}},
}

\tikzset{
  self naming node/.style={
    node contents/.forward to=/cleanser/input,
    /cleanser/output/.forward to=/tikz/name,},
}

\def\kDMeanCut#1:->#2\into#3\GO%
  {\kDCSDef{#3}{#2}}

\def\kDMeaningOf#1\into#2\GO%
  {\def\foo{#1}\edef\bar{\meaning\foo}%
   \expandafter\kDMeanCut\bar\into#2\GO}

\def\kDSanitize#1\into#2\GO%
  {\kDMeaningOf#1 \into Meaning\GO% I put an extra space for easy parsing
   \expandafter\edef\csname#2\endcsname% and catch it as the empty exception
     {\ifx\Meaning\space\else\expandafter\kDSieveString\Meaning\GO\fi}}

% Maybe a switch to preserve/erase spaces would be useful here
\def\kDSieveString#1 #2\GO%
  {\kDSieveWord#1\GO\ifx\relax#2\else\space\kDSieveString#2\GO\fi}

\def\kDSieveWord#1#2\GO%
  {\kDSieveCharacter#1\GO\ifx\relax#2\else\kDSieveWord#2\GO\fi}

% Spaces after number to avoid stray \relax
\def\kDSieveCharacter#1\GO%
  {\ifnum`#1=36 \else% $ dollar
   \ifnum`#1=44 \else% , comma
   \ifnum`#1=46 \else% . dot
   \ifnum`#1=58 \else% : colon
   \ifnum`#1=92 \else% \ backslash
   #1\fi\fi\fi\fi\fi}

% #################################################################### THE END #

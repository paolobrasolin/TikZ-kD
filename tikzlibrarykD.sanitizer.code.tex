% tikzlibrarykD.code.tex
%
% Copyright 2015 by Paolo Brasolin <paolo.brasolin@gmail.com>
%
% This program is free software: you can redistribute it and/or modify
% it under the terms of the GNU General Public License as published by
% the Free Software Foundation, either version 3 of the License, or
% (at your option) any later version.
%
% This program is distributed in the hope that it will be useful,
% but WITHOUT ANY WARRANTY; without even the implied warranty of
% MERCHANTABILITY or FITNESS FOR A PARTICULAR PURPOSE.  See the
% GNU General Public License for more details.
% 
% You should have received a copy of the GNU General Public License
% along with this program.  If not, see <http://www.gnu.org/licenses/>.

% ################################################################ LET'S DANCE #

\newif\ifnamegeneratorexpands
\pgfkeys{
  /tikz/name generator/expand/.is if=namegeneratorexpands,
  /tikz/name generator/.code={
    \ifnamegeneratorexpands\edef\tmp{#1}\else\def\tmp{#1}\fi
    \expandafter\kDSanitize\tmp\into SafeName\GO
    \edef\tmp{\SafeName}
    \message{New safename: from "#1" to "\tmp".}
    \pgfkeys{/tikz/name/.expanded={\SafeName}}
},
  /tikz/self naming node/.style={
    /tikz/node contents/.forward to=/tikz/name generator},
}

\def\kDSanitize#1\into#2\GO%
  {\kDMeaningOf#1 \into Meaning\GO% Note I put an extra space for easy parsing
   \kDCSDef{#2}% and catch it as the empty exception
     {\ifx\Meaning\space\else\expandafter\kDSieveString\Meaning\GO\fi}%
}

\def\kDMeaningOf#1\into#2\GO%
  {\def\tmp{#1}\edef\zop{\meaning\tmp}%\show\zop% \meaning#1 instead of \zopt @nexline
   \expandafter\kDSplice\zop\at>\into Trash\and #2\GO}

% Note: maybe a switch to preserve/erase spaces would be useful
\def\kDSieveString#1 #2\GO%
  {\kDSieveWord#1\GO\ifx\relax#2\else\space\kDSieveString#2\GO\fi}

\def\kDSieveWord#1#2\GO%
  {\kDSieveCharacter#1\GO\ifx\relax#2\else\kDSieveWord#2\GO\fi}

% Note: spaces after number avoid stray \relax
\def\kDSieveCharacter#1\GO%
  {%
   \ifnum`#1=36 \else% $ dollar
   \ifnum`#1=44 \else% , comma
   \ifnum`#1=46 \else% . dot
   \ifnum`#1=58 \else% : colon
   \ifnum`#1=92 \else% \ backslash
   #1\fi\fi\fi\fi\fi}

% #################################################################### THE END #

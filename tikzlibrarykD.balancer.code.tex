% tikzlibrarykD.balancer.code.tex
%
% Copyright 2015 by Paolo Brasolin <paolo.brasolin@gmail.com>
%
% This program is free software: you can redistribute it and/or modify
% it under the terms of the GNU General Public License as published by
% the Free Software Foundation, either version 3 of the License, or
% (at your option) any later version.
%
% This program is distributed in the hope that it will be useful,
% but WITHOUT ANY WARRANTY; without even the implied warranty of
% MERCHANTABILITY or FITNESS FOR A PARTICULAR PURPOSE.  See the
% GNU General Public License for more details.
% 
% You should have received a copy of the GNU General Public License
% along with this program.  If not, see <http://www.gnu.org/licenses/>.

% ######################################################################## SOF #

% There macros are the primitives to handle length partitioning.
% Rigid lengths are either a dimension or a number representing a fraction
% of the Total.  Loose lengths are starred numbers that proportionally fill
% the rest.  So, a specification like
%     0|0.1|*|*3|1cm    with Total=10cm
% would ultimately be parsed as
%     0cm|1cm|2cm|6cm|1cm
% These macros here parse just single numbers, but standard gobbling procedure
% can easily generalize the parsing to a string like the example.

% Basic initialization
\def\kDBalancerInit#1%
  {\def\kDBalancerLoose{0}%
   \def\kDBalancerRigid{0}%
   \def\kDBalancerTotal{#1}}

% First we tally the totals
\def\kDBalancerTallyLoose*#1\to#2\GO%
  {\ifx\relax#1\relax\pgfmathparse{1}\else\pgfmathparse{#1}\fi%
   \expandafter\edef\csname#2\endcsname{*\pgfmathresult}%
   \pgfmathsetmacro\kDBalancerLoose{\kDBalancerLoose+\pgfmathresult}}
\def\kDBalancerTallyRigid#1\to#2\GO%
  {\ifx\relax#1\relax\pgfmathparse{0}\else\pgfmathparse{#1}\fi%
   \ifpgfmathunitsdeclared\pgfmathdivide{#1}{\kDBalancerTotal}\fi%
   \expandafter\pgfmathsetmacro\csname#2\endcsname{\pgfmathresult}%
   \pgfmathsetlengthmacro\kDBalancerRigid{\kDBalancerRigid+\pgfmathresult}}
\kDDefStarred kDBalancerTally%
  \with\kDBalancerTallyLoose%
   \and\kDBalancerTallyRigid\GO

% Then we can produce the actual lengths
\def\kDBalancerGaugeLoose*#1\to#2\GO%
  {\expandafter\pgfmathsetmacro\csname#2\endcsname%
     {#1*(1-\kDBalancerRigid)/\kDBalancerLoose}}
\def\kDBalancerGaugeRigid#1\to#2\GO%
  {\expandafter\pgfmathsetmacro\csname#2\endcsname{#1}}
\kDDefStarred kDBalancerGauge%
  \with\kDBalancerGaugeLoose%
   \and\kDBalancerGaugeRigid\GO

% ######################################################################## EOF #

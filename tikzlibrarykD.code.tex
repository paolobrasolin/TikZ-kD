% tikzlibrarykD.code.tex
%
% Copyright 2015 by Paolo Brasolin <paolo.brasolin@gmail.com>
%
% This program is free software: you can redistribute it and/or modify
% it under the terms of the GNU General Public License as published by
% the Free Software Foundation, either version 3 of the License, or
% (at your option) any later version.
%
% This program is distributed in the hope that it will be useful,
% but WITHOUT ANY WARRANTY; without even the implied warranty of
% MERCHANTABILITY or FITNESS FOR A PARTICULAR PURPOSE.  See the
% GNU General Public License for more details.
% 
% You should have received a copy of the GNU General Public License
% along with this program.  If not, see <http://www.gnu.org/licenses/>.

% ################################################################ LET'S DANCE #


% ==================================================================== STYLING =

\def\kDtikzset{\pgfqkeys{/tikz/kD}}

\kDtikzset{
%  arrows/.code={\tikzcdset{every arrow/.append style={#1}}},
%  labels/.code={\tikzcdset{every label/.append style={#1}}},
%  cells/.code={\tikzcdset{every cell/.append style={#1}}},
%  diagrams/.code={\tikzcdset{every diagram/.append style={#1}}},
%  execute before arrows/.code={\expandafter\def\expandafter\tikzcd@before@paths@hook\expandafter{\tikzcd@before@paths@hook#1}},
  crossing over/.style={
    /tikz/preaction={
      /tikz/draw=\pgfkeysvalueof{/tikz/kD/background color},
      /tikz/arrows=-,
      /tikz/line width=\pgfkeysvalueof{/tikz/kD/crossing over clearance}}},
%  cramped/.code={\tikzcdset{
%    every matrix/.append style={inner sep=+-0.3em},
%    every cell/.append style={inner sep=+0.3em}}},
%  row sep/.code={\tikzcd@sep{row}{#1}},
%  column sep/.code={\tikzcd@sep{column}{#1}},
%  sep/.code={\tikzcdset{row sep={#1},column sep={#1}}},
}


% Default settings
\kDtikzset{
  every arrow/.style={
%    /tikz/draw,
%    /tikz/line width=rule_thickness,
%    /tikz/commutative diagrams/rightarrow
},
  every label/.style={
/tikz/auto,
%    /tikz/font=\everymath\expandafter{\the\everymath\scriptstyle},
%/tikz/inner sep=+0.5ex
},
  every cell/.style={
%    /tikz/shape={asymmetrical rectangle},
%    /tikz/inner xsep=+1ex,
%    /tikz/inner ysep=+0.85ex
},
  every matrix/.style={
/tikz/inner sep=0.618em,
/tikz/row sep=1.618em,
/tikz/column sep=2.618em
},
  every diagram/.style={
%    /tikz/commutative diagrams/row sep=normal,
%    /tikz/commutative diagrams/column sep=normal,
%    /tikz/baseline=+0pt
},
  crossing over clearance/.initial=0.236em,
  background color/.initial=white
}


% ======================================================== NODE NAME GENERATOR =

\def\kDSanitize#1\into#2\GO%
  {\kDMeaningOf#1 \into Meaning\GO% Note I put an extra space for easy parsing
   \expandafter\def\csname#2\endcsname% and catch it as the empty exception
     {\ifx\Meaning\space\else\expandafter\kDSieveString\Meaning\GO\fi}}

\def\kDMeaningOf#1\into#2\GO%
  {\expandafter\CutHeadOf\meaning#1\at>\into Trash\and #2\GO}

% Note: maybe a switch to preserve/erase spaces would be useful
\def\kDSieveString#1 #2\GO%
  {\kDSieveWord#1\GO\ifx\relax#2\else\space\kDSieveString#2\GO\fi}

\def\kDSieveWord#1#2\GO%
  {\kDSieveCharacter#1\ifx\relax#2\else\kDSieveWord#2\GO\fi}

% Note: spaces after number avoid stray \relax
\def\kDSieveCharacter#1%
  {\ifnum`#1=44 \else% , comma
   \ifnum`#1=46 \else% . dot
   \ifnum`#1=58 \else% : colon
   \ifnum`#1=92 \else% \ backslash
   #1\fi\fi\fi\fi}

% ========================================================= DRAWING PRIMITIVES =

\def\DrawObject#1\withNode#2\GO%
  {\kDSanitize#1\into SafeName\GO%
   \edef\tmp{[name={\SafeName}] #2 {$#1$}}%
   \expandafter\node\tmp;}

\def\DrawMorphism#1\withNode#2\withEdge#3\from#4\to#5\GO%
  {%\expandafter\kDSanitize#1\into SafeName\GO%
   \edef\tmp{[kD/every arrow] (#4) edge [#3]
    node [kD/every label,name={}] #2 {$#1$} (#5)}%
    \expandafter\path\tmp;}

% ============================================================ PARSING HELPERS =

\def\kDCSDef#1#2{\expandafter\def\csname#1\endcsname{#2}}

\def\kDSplice#1\at#2\into#3\and#4\GO%
  {\def\Rimmer##1#2\GO{\kDCSDef{#4}{##1}}%
   \def\Cutter##1#2##2\GO%
     {\ifx##2\relax\relax\kDCSDef{#3}{}\kDCSDef{#4}{##1}%
      \else\kDCSDef{#3}{##1}\Rimmer##2\GO\fi}%
   \Cutter#1#2\GO}

%\def\kDSplice#1\atFirst#2\into#3\and#4\GO%
%  {\def\Rimmer##1#2\GO{##1}%
%   \def\Cutter##1#2##2\GO%
%     {\expandafter\edef\csname#3\endcsname{##1}%
%      \expandafter\edef\csname#4\endcsname%
%        {\ifx##2\relax\else\Rimmer##2\GO\fi}}%
%   \expandafter\Cutter#1#2\GO}
%
\def\CutHeadOf#1\at#2\into#3\and#4\GO%
  {\def\Cutter##1#2##2\GO%
     {\expandafter\def\csname#3\endcsname{##1}%
      \expandafter\def\csname#4\endcsname{##2}}%
   \Cutter#1\GO}
%
%\def\Flip#1\aroundFirst#2\into#3\GO%
%  {\def\Flipper#2##1#2##2\GO%
%     {\expandafter\def\csname#3\endcsname{##2##1}}%
%   \def\flippable{#2#1#2}%   BIG PROBLEM HERE
%   \expandafter\Flipper\flippable\GO}
%
%\def\CutTailOf#1\at#2\into#3\and#4\GO%
%  {\Flip#1\aroundFirst#2\into Flipped\GO%
%   \CutHeadOf\Flipped#2\at#2\into #4\and #3\GO%
%   \expandafter\CutHeadOf\csname#3\endcsname#2\at#2\into #3\and Trash\GO}

% ==================================================================== PARSERS =

\def\ParseMorphism#1\GO%
  {\kDSplice#1\at:\into Head\and EdgeSpec\GO%
   \expandafter\kDSplice\Head\at=\into NodeSpec\and MathSpec\GO}

\def\ParseMorphismAlt#1\GO%
  {\kDSplice#1\at=\into NodeSpec\and MathSpec\GO}

\def\ParseObject#1\GO%
  {\kDSplice#1\at=\into NodeSpec\and MathSpec\GO}

% ========================================================== INTERNAL COMMANDS =

\def\DoObject#1\GO%
  {\ParseObject#1\GO
   \DrawObject\MathSpec\withNode\NodeSpec\GO}

\def\DoMorphism #1 #2 #3\GO%
  {\ParseMorphism#2\GO%
   \DrawMorphism\MathSpec\withNode\NodeSpec\withEdge{\EdgeSpec}\from#1\to#3\GO}

\def\DoMorphismChain #1 #2 #3 #4\GO%
  {\DoMorphism {#1} {#2} {#3}\GO%
   \ifx\relax#4\else\DoMorphismChain {#3} #4\GO\fi}

\def\DoMorphismAlt#1: #2 #3 #4\GO%
  {\ParseMorphismAlt#1\GO%
   \DrawMorphism\MathSpec\withNode\NodeSpec\withEdge{#3}\from#2\to#4\GO}

\def\DoMorphismAltList #1, #2\GO%
  {\DoMorphismAlt#1\GO%
   \ifx\relax#2\else\DoMorphismAltList #2\GO\fi}

% ============================================================= USER INTERFACE =

\def\obj#1;{\DoObject #1\GO}

\def\MorChain #1;{\DoMorphismChain #1 \GO}
\def\MorList* #1;{\DoMorphismAltList #1, \GO}


\def\mor{\futurelet\MaybeStar\MaybeStarredMor}
\def\MaybeStarredMor{\ifx*\MaybeStar \let\next\MorList
\else \let\next\MorChain \fi \next}

% ================================================================= ENVIROMENT =

\def\startcommutativediagram%
  {\starttikzpicture[]}
\def\stopcommutativediagram%
  {\stoptikzpicture}

% #################################################################### THE END #

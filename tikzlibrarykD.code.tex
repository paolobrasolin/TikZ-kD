% tikzlibrarykD.code.tex
%
% Copyright 2015 by Paolo Brasolin <paolo.brasolin@gmail.com>
%
% This program is free software: you can redistribute it and/or modify
% it under the terms of the GNU General Public License as published by
% the Free Software Foundation, either version 3 of the License, or
% (at your option) any later version.
%
% This program is distributed in the hope that it will be useful,
% but WITHOUT ANY WARRANTY; without even the implied warranty of
% MERCHANTABILITY or FITNESS FOR A PARTICULAR PURPOSE.  See the
% GNU General Public License for more details.
% 
% You should have received a copy of the GNU General Public License
% along with this program.  If not, see <http://www.gnu.org/licenses/>.

% ################################################################ LET'S DANCE #


% ==================================================================== STYLING =

\def\kDSet{\pgfqkeys{/tikz/kD}}

\usetikzlibrary{decorations.pathreplacing}

\tikzset{
  kD/.search also={/tikz},
  kD/.cd,
  /handlers/first char syntax=true,
  /handlers/first char syntax/the character (/.initial=\kDNamingShortcut,
  shift/.style={transform canvas=
    {shift={($(\tikztostart)!#1!-90:(\tikztotarget)-(\tikztostart)$)}}},
  slide/.style={transform canvas=
    {shift={($(\tikztostart)!#1!0:(\tikztotarget)-(\tikztostart)$)}}},
  chop/.style args={#1-#2}{ 
    decoration={ 
      show path construction, 
      curveto code={ 
        \pgfpathcurvebetweentime{#1}{#2}%
        {\pgfpointdecoratedinputsegmentfirst}% 
        {\pgfpointdecoratedinputsegmentsupporta}% 
        {\pgfpointdecoratedinputsegmentsupportb}% 
        {\pgfpointdecoratedinputsegmentlast}% 
      },
      lineto code={ % kinda raw, this one
        \pgfpathcurvebetweentime{#1}{#2}%
        {\pgfpointdecoratedinputsegmentfirst}% 
        {\pgfpointdecoratedinputsegmentfirst}% 
        {\pgfpointdecoratedinputsegmentlast}% 
        {\pgfpointdecoratedinputsegmentlast}% 
      }
    }, 
    decorate 
  } 
}


\kDSet{
  ÷/.style={crossing over},
  ^>/.style={bend left,->},
  ÷>/.style={÷,->},
  </.style={near start},
  >/.style={near end},
}

\def\kDRemoveParens(#1){#1}
\def\kDNamingShortcut#1{\pgfkeysalso{/tikz/name={\kDRemoveParens#1}}}

\kDSet{
%  arrows/.code={\tikzcdset{every arrow/.append style={#1}}},
%  labels/.code={\tikzcdset{every label/.append style={#1}}},
%  cells/.code={\tikzcdset{every cell/.append style={#1}}},
%  diagrams/.code={\tikzcdset{every diagram/.append style={#1}}},
%  execute before arrows/.code={\expandafter\def\expandafter\tikzcd@before@paths@hook\expandafter{\tikzcd@before@paths@hook#1}},
  crossing over/.style={
    /tikz/preaction={
      /tikz/draw=\pgfkeysvalueof{/tikz/kD/background color},
      /tikz/arrows=-,
      /tikz/line width=\pgfkeysvalueof{/tikz/kD/crossing over clearance}}},
%  cramped/.code={\tikzcdset{
%    every matrix/.append style={inner sep=+-0.3em},
%    every cell/.append style={inner sep=+0.3em}}},
%  row sep/.code={\tikzcd@sep{row}{#1}},
%  column sep/.code={\tikzcd@sep{column}{#1}},
%  sep/.code={\tikzcdset{row sep={#1},column sep={#1}}},
}


% Default settings
\kDSet{
  every object/.style={
%    /tikz/draw,
%    /tikz/line width=rule_thickness,
%    /tikz/commutative diagrams/rightarrow
},
  every arrow/.style={
%    /tikz/draw,
%    /tikz/line width=rule_thickness,
%    /tikz/commutative diagrams/rightarrow
},
  every label/.style={
/tikz/auto,
%    /tikz/font=\everymath\expandafter{\the\everymath\scriptstyle},
%/tikz/inner sep=+0.5ex
},
  every cell/.style={
%    /tikz/shape={},
%    /tikz/inner xsep=+1ex,
%    /tikz/inner ysep=+0.85ex
},
  every matrix/.style={
    /tikz/inner sep=0.618em,
    /tikz/row sep=1.618em,
    /tikz/column sep=2.618em
},
  every diagram/.style={
%    /tikz/commutative diagrams/row sep=normal,
%    /tikz/commutative diagrams/column sep=normal,
%    /tikz/baseline=+0pt
},
  crossing over clearance/.initial=0.236em,
  background color/.initial=white
}


% ======================================================== NODE NAME GENERATOR =

\def\kDSanitize#1\into#2\GO%
  {\kDMeaningOf#1 \into Meaning\GO% Note I put an extra space for easy parsing
   \expandafter\def\csname#2\endcsname% and catch it as the empty exception
     {\ifx\Meaning\space\else\expandafter\kDSieveString\Meaning\GO\fi}}

\def\kDMeaningOf#1\into#2\GO%
  {\expandafter\kDSplice\meaning#1\at>\into Trash\and #2\GO}

% Note: maybe a switch to preserve/erase spaces would be useful
\def\kDSieveString#1 #2\GO%
  {\kDSieveWord#1\GO\ifx\relax#2\else\space\kDSieveString#2\GO\fi}

\def\kDSieveWord#1#2\GO%
  {\kDSieveCharacter#1\ifx\relax#2\else\kDSieveWord#2\GO\fi}

% Note: spaces after number avoid stray \relax
\def\kDSieveCharacter#1%
  {\ifnum`#1=44 \else% , comma
   \ifnum`#1=46 \else% . dot
   \ifnum`#1=58 \else% : colon
   \ifnum`#1=92 \else% \ backslash
   #1\fi\fi\fi\fi}

% ========================================================= DRAWING PRIMITIVES =

\def\DrawObject#1\withNode#2\GO%
  {\kDSanitize#1\into SafeName\GO%
   \edef\tmp{[kD/.cd,every object,(\SafeName),#2] {$#1$}}%
   \expandafter\node\tmp;}

\def\DrawMorphism#1\withNode#2\withEdge#3\from#4\to#5\GO%
  {\kDSanitize#1\into SafeName\GO%
   \edef\tmp{(#4) edge [kD/.cd,every arrow,#3]
    node [kD/.cd,every label,(\SafeName),#2] {$#1$} (#5)}%
    \expandafter\path\tmp;}

% ============================================================ GENERAL HELPERS =

\def\kDCSDef#1#2{\expandafter\def\csname#1\endcsname{#2}}

% ============================================================ PARSING HELPERS =

\def\kDSplice#1\at#2\into#3\and#4\GO%
  {\def\Rimmer##1#2\GO{\kDCSDef{#4}{##1}}%
   \def\Cutter##1#2##2\GO%
     {\def\snd{##2}\ifx\snd\empty\kDCSDef{#3}{}\kDCSDef{#4}{##1}%
      \else\kDCSDef{#3}{##1}\Rimmer##2\GO\fi}%
   \Cutter#1#2\GO}

\def\kDTrisect#1\at#2\into#3\and#4\and#5\GO%
  {\def\Cut#2##1#2##2#2##3#2##4\GO%
     {\def\cnt{##4}\def\one{#2}\def\two{#2#2}\def\thr{#2#2#2}
      \ifx\cnt\one\kDCSDef{#3}{##1}\kDCSDef{#4}{}   \kDCSDef{#5}{}   \else
      \ifx\cnt\two\kDCSDef{#3}{##2}\kDCSDef{#4}{##1}\kDCSDef{#5}{}   \else
      \ifx\cnt\thr\kDCSDef{#3}{##3}\kDCSDef{#4}{##2}\kDCSDef{#5}{##1}\else
      \errmessage{PROBLEMS, YO!}\fi\fi\fi}
   \Cut#2#1#2#2#2#2\GO}

\def\kDBisect#1\at#2\into#3\and#4\GO%
  {\def\Cut#2##1#2##2#2##3\GO%
     {\def\cnt{##3}\def\one{#2}\def\two{#2#2}
      \ifx\cnt\one\kDCSDef{#3}{##1}\kDCSDef{#4}{}   \else
      \ifx\cnt\two\kDCSDef{#3}{##2}\kDCSDef{#4}{##1}\else
      \errmessage{PROBLEMS, YO!}\fi\fi}
   \Cut#2#1#2#2#2\GO}

% ========================================================== INTERNAL COMMANDS =

\def\DoObject#1\GO%
  {\kDBisect#1\at:\into MathSpec\and NodeSpec\GO%
   \DrawObject\MathSpec\withNode\NodeSpec\GO}

\def\DoMorphism #1 #2 #3\GO%
  {\kDTrisect#2\at:\into EdgeSpec\and MathSpec\and NodeSpec\GO%
   \DrawMorphism\MathSpec\withNode\NodeSpec\withEdge\EdgeSpec\from#1\to#3\GO}

\let\kDThisSource\relax \let\kDLastSource\relax
\let\kDThisTarget\relax \let\kDLastTarget\relax
\def\kDLast{*}

\def\DoMorphismChain #1 #2 #3 #4\GO%
  {\def\kDSource{#1}\ifx\kDSource\kDLast\let\kDSource\kDLastSource\fi
   \def\kDTarget{#3}\ifx\kDTarget\kDLast\let\kDTarget\kDLastTarget\fi
   \DoMorphism {\kDSource} {#2} {\kDTarget}\GO%
   \ifx\kDThisSource\relax\let\kDThisSource\kDSource\fi
   \ifx\relax#4\relax
   \let\kDLastSource\kDThisSource\let\kDThisSource\relax
   \let\kDLastTarget\kDTarget
   \else\DoMorphismChain {#3} #4\GO\fi}

% ============================================================= USER INTERFACE =

\def\obj#1;{\DoObject #1\GO}

\def\MorChain #1;{\DoMorphismChain #1 \GO}

%\def\MorOpt[#1] #2;{\DoMorphismChain #2, \GO}
%
%\def\mor{\let\kDCommonOpt\relax\futurelet\MaybeSquare\MaybeOptMor}
%\def\MaybeOptMor{\ifx[\MaybeSquare \let\next\MorOpt
%\else \let\next\MorNoOpt \fi \next}

\let\mor\MorChain

% ================================================================= ENVIROMENT =

% (La)TeX
\def\kD{\tikzpicture}
\def\endkD{\endtikzpicture}
\let\commutativediagram\kD
\let\endcommutativediagram\endkD

% ConTeXt
\def\startkD{\starttikzpicture}
\def\stopkD{\stoptikzpicture}
\let\startcommutativediagram\startkD
\let\stopcommutativediagram\stopkD

% #################################################################### THE END #

% tikzlibrarykD.code.tex
%
% Copyright 2015 by Paolo Brasolin <paolo.brasolin@gmail.com>
%
% This program is free software: you can redistribute it and/or modify
% it under the terms of the GNU General Public License as published by
% the Free Software Foundation, either version 3 of the License, or
% (at your option) any later version.
%
% This program is distributed in the hope that it will be useful,
% but WITHOUT ANY WARRANTY; without even the implied warranty of
% MERCHANTABILITY or FITNESS FOR A PARTICULAR PURPOSE.  See the
% GNU General Public License for more details.
% 
% You should have received a copy of the GNU General Public License
% along with this program.  If not, see <http://www.gnu.org/licenses/>.

% ################################################################ LET'S DANCE #

% ==================================================================== STYLING =

\def\kDRemoveParens(#1){#1}
\def\kDNamingShortcut#1{\pgfkeysalso{/tikz/name={\kDRemoveParens#1}}}

% Default settings
\pgfqkeys{/kD}{
  .search also=/tikz,
  objects/.search also=/tikz,
  arrows/.search also=/tikz,
  labels/.search also=/tikz,
  matrices/.search also=/tikz,
  every object/.style={/kD/objects/.cd},
  every arrow/.style={/kD/arrows/.cd},
  every label/.style={/kD/labels/.cd},
  every matrix/.style={/kD/matrices/.cd},
%%%%%%%%%%%%%%%%%%%%%%%%%%%%%%%%%%%%%%%%%%%%%%%%%%%%%%%%%%%%%%%%%%%%%%%%%%%%%%%%
  current chain/.style={},
  current chain/arrows/.style={},
  current chain/labels/.style={},
%%%%%%%%%%%%%%%%%%%%%%%%%%%%%%%%%%%%%%%%%%%%%%%%%%%%%%%%%%%%%%%%%%%%%%%%%%%%%%%%
  every object/.append style={
    self naming node,},
  every arrow/.append style={
    /kD/current chain/arrows,},
  every label/.append style={
    self naming node,
    auto,
    font=\everymath\expandafter{\the\everymath\scriptstyle},
    inner sep=+0.5ex,
    /kD/current chain/labels,},
%%%%%%%%%%%%%%%%%%%%%%%%%%%%%%%%%%%%%%%%%%%%%%%%%%%%%%%%%%%%%%%%%%%%%%%%%%%%%%%%
  /kD/arrows/.cd,
    ->/.style={-Stealth},
  /kD/matrices/.cd,
    comb/.style={
      /tikz/row sep={sqrt(3/4)*#1,between origins},
      /tikz/column sep={#1,between origins},
      /tikz/every even row/.style={xshift=-0.5*#1},},
    comb/.default=4em,
    mesh/.style={
      /tikz/row sep={#1,between origins},
      /tikz/column sep={#1,between origins},},
    mesh/.default=4em,
    gold/.style={
      /tikz/row sep=#1,
      /tikz/column sep=1.618*#1,},
    gold/.default=1.618em,
  /kD/arrows/.cd,
    crossing over/.style={preaction={-,
      draw=\pgfkeysvalueof{/kD/arrows/crossing over/color},
      line width=\pgfkeysvalueof{/kD/arrows/crossing over/clearance},},},
    crossing over/clearance/.initial=0.236em,
    crossing over/color/.initial=white,
    ÷/.style={crossing over},
    ÷>/.style={÷,->},
    shift/.style={transform canvas=
      {shift={($(\tikztostart)!#1!-90:(\tikztotarget)-(\tikztostart)$)}}},
    slide/.style={transform canvas=
      {shift={($(\tikztostart)!#1!0:(\tikztotarget)-(\tikztostart)$)}}},
  /kD/labels/.cd,
    mid/.style={inner sep=1.5pt,shape=circle,fill=white,anchor=center},
    </.style={near start},
    >/.style={near end},
%%%%%%%%%%%%%%%%%%%%%%%%%%%%%%%%%%%%%%%%%%%%%%%%%%%%%%%%%%%%%%%%%%%%%%%%%%%%%%%%
  /handlers/first char syntax=true,
  /handlers/first char syntax/the character (/.initial=\kDNamingShortcut,
}

% =========================================================== arrow keys: CHOP =

\def\kDChopParse#1|#2|#3\GO%
  {\kDBalancerInit\pgfdecoratedpathlength
   \kDBalancerTally#1\to One\GO
   \kDBalancerTally#2\to Two\GO
   \kDBalancerTally#3\to Thr\GO
   \expandafter\kDBalancerGauge\One\to Fst\GO
   \expandafter\kDBalancerGauge\Thr\to Lst\GO
   \pgfmathsetmacro\start{\Fst}
   \pgfmathsetmacro\stop{1-\Lst}}

\tikzset{kD/chop/.style={ 
  decoration={ 
    show path construction, 
    curveto code={
      \kDFullExpandAfter\kDChopParse{#1}\GO
      \pgfpathcurvebetweentime{\start}{\stop}
      {\pgfpointdecoratedinputsegmentfirst}
      {\pgfpointdecoratedinputsegmentsupporta}
      {\pgfpointdecoratedinputsegmentsupportb}
      {\pgfpointdecoratedinputsegmentlast}},
    lineto code={ % kinda raw, this one
      \kDFullExpandAfter\kDChopParse{#1}\GO
      \pgfpathcurvebetweentime{\start}{\stop}
      {\pgfpointdecoratedinputsegmentfirst}
      {\pgfpointdecoratedinputsegmentfirst}
      {\pgfpointdecoratedinputsegmentlast}
      {\pgfpointdecoratedinputsegmentlast}}
  },decorate}}

% ============================================================ PARSING HELPERS =

\def\kDSplice#1\at#2\into#3\and#4\GO%
  {\def\Rimmer##1#2\GO{\kDCSDef{#4}{##1}}%
   \def\Cutter##1#2##2\GO%
     {\def\snd{##2}\ifx\snd\empty\kDCSDef{#3}{}\kDCSDef{#4}{##1}%
      \else\kDCSDef{#3}{##1}\Rimmer##2\GO\fi}%
   \Cutter#1#2\GO}

\def\kDTrisect#1\at#2\into#3\and#4\and#5\GO%
  {\def\Cut#2##1#2##2#2##3#2##4\GO%
     {\def\cnt{##4}\def\one{#2}\def\two{#2#2}\def\thr{#2#2#2}
      \ifx\cnt\one\kDCSDef{#3}{##1}\kDCSDef{#4}{}   \kDCSDef{#5}{}   \else
      \ifx\cnt\two\kDCSDef{#3}{##2}\kDCSDef{#4}{##1}\kDCSDef{#5}{}   \else
      \ifx\cnt\thr\kDCSDef{#3}{##3}\kDCSDef{#4}{##2}\kDCSDef{#5}{##1}\else
      \errmessage{PROBLEMS, YO!}\fi\fi\fi}
   \Cut#2#1#2#2#2#2\GO}

\def\kDBisect#1\at#2\into#3\and#4\GO%
  {\def\Cut#2##1#2##2#2##3\GO%
     {\def\cnt{##3}\def\one{#2}\def\two{#2#2}
      \ifx\cnt\one\kDCSDef{#3}{##1}\kDCSDef{#4}{}   \else
      \ifx\cnt\two\kDCSDef{#3}{##2}\kDCSDef{#4}{##1}\else
      \errmessage{PROBLEMS, YO!}\fi\fi}
   \Cut#2#1#2#2#2\GO}

% ========================================================= DRAWING PRIMITIVES =

\def\kDObject#1\withNode#2\GO%
  {\node[/kD/every object, node contents={$#1$}, #2];}

\def\kDMorphism#1\withNode#2\withEdge#3\from#4\to#5\GO%
  {\path (#4) edge [/kD/every arrow, #3, edge node={
              node [/kD/every label, node contents={$#1$}, #2]}] (#5);}

% ========================================================== PARAMETER PARSING =

\def\kDDoObject#1\GO%
  {\kDBisect#1\at:\into MathSpec\and NodeSpec\GO%
   \expandafter\expandafter\expandafter\kDObject%
   \expandafter\MathSpec\expandafter\withNode\NodeSpec\GO}

\def\kDDoMorphism#1 #2 #3\GO%
  {\kDTrisect#2\at:\into EdgeSpec\and MathSpec\and NodeSpec\GO%
   \def\tmp{\NodeSpec,\kDChainOptions} % ------------------------------------------ WARNING, THIS WON'T WORK
   \expandafter\expandafter\expandafter%
   \expandafter\expandafter\expandafter%
   \expandafter\kDMorphism\expandafter%
   \expandafter\expandafter\MathSpec%
   \expandafter\expandafter\expandafter%
   \withNode\expandafter\NodeSpec%
   \expandafter\withEdge\EdgeSpec\from#1\to#3\GO}

\let\kDThisSource\relax \let\kDLastSource\relax
\let\kDThisTarget\relax \let\kDLastTarget\relax
\def\kDLast{*}

\def\kDDoMorphismChain#1 #2 #3 #4\GO%
  {\def\kDSource{#1}\ifx\kDSource\kDLast\let\kDSource\kDLastSource\fi
   \def\kDTarget{#3}\ifx\kDTarget\kDLast\let\kDTarget\kDLastTarget\fi
   \kDDoMorphism{\kDSource} {#2} {\kDTarget}\GO%
   \ifx\kDThisSource\relax\let\kDThisSource\kDSource\fi
   \ifx\relax#4\relax
   \let\kDLastSource\kDThisSource\let\kDThisSource\relax
   \let\kDLastTarget\kDTarget
   \else\kDDoMorphismChain{#3} #4\GO\fi}

% ============================================================= USER INTERFACE =

\def\obj#1;%
  {\kDDoObject#1\GO}

\def\kDChainOptions{}

\def\MorChain #1;{%
  \pgfkeys{/kD/current chain/arrows/.style={}}%
  \pgfkeys{/kD/current chain/labels/.style={}}%
  \kDDoMorphismChain #1 \GO}
\def\MorOpt[#1] #2;{%
  \kDBisect#1\at:\into ArrowSpec\and LabelSpec\GO%
  \pgfkeys{/kD/current chain/arrows/.estyle=\ArrowSpec}%
  \pgfkeys{/kD/current chain/labels/.estyle=\LabelSpec}%
  \kDDoMorphismChain #2 \GO}

\def\mor{\let\kDCommonOpt\relax\futurelet\MaybeSquare\MaybeOptMor}
\def\MaybeOptMor{\ifx[\MaybeSquare \let\next\MorOpt
\else \let\next\MorChain \fi \next}



\def\lattice[#1]{\matrix[/kD/every matrix,#1]}


% #################################################################### THE END #

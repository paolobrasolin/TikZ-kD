% manual.tex
%
% Copyright 2015 by Paolo Brasolin <paolo.brasolin@gmail.com>
%
% This program is free software: you can redistribute it and/or modify
% it under the terms of the GNU General Public License as published by
% the Free Software Foundation, either version 3 of the License, or
% (at your option) any later version.
%
% This program is distributed in the hope that it will be useful,
% but WITHOUT ANY WARRANTY; without even the implied warranty of
% MERCHANTABILITY or FITNESS FOR A PARTICULAR PURPOSE.  See the
% GNU General Public License for more details.
% 
% You should have received a copy of the GNU General Public License
% along with this program.  If not, see <http://www.gnu.org/licenses/>.

% ######################################################################## SOF #

% ############################################################ REQUIRE CONTEXT #

\newif\ifConTeXt

% Trick stolen from iftex. The second line is expanded inside the group so
% the global scope isn't polluted by \csname defining the token.
\begingroup\expandafter\expandafter\expandafter\endgroup\expandafter
  \ifx\csname starttext\endcsname\relax\ConTeXtfalse\else\ConTeXttrue\fi

\def\RequireConTeXt%
  {\ifConTeXt\else\begingroup
     \errorcontextlines=-1\relax
     \newlinechar=10\relax
     \let~\space
     \errmessage{^^J^^J
~~~~~_________________________^^J
~~~~~|~~~~~~~~~~~~~~~~~~~~~~~|^^J
~~~~~|~~ ConTeXt (MKIV) is ~~|^^J
~~~~~|~ required to compile ~|^^J
~~~~~|~ this document. ~~~~~~|^^J
~~~~~|~~~~~~~~~~~~~~~~~~~~~~~|^^J
~~~~~|~~~~~~~~~~~~~~~~~ -P. ~|^^J
~~~~~|~~~~~~~~~~~~~~~~~~~~~~~|^^J
~~~~~|_______________________|^^J^^J^^J}%
   \endgroup\fi}

\RequireConTeXt

% ##############################################################################

\usemodule[tikz]
\usetikzlibrary[calc]
\usetikzlibrary[positioning]

\usetikzlibrary[kD]

% ==============================================================================

\setuplayout
  [grid=yes,
   backspace=0.125\paperwidth,width=middle,
   topspace=0.088\paperheight,height=middle,
   header=0bp,footer=0bp]
\setuphead[title][alternative=margin,align=flushright,page=no]
\setuppagenumbering[location=]
\setuptyping[option=TEX,style=tx]
\setupframed[width=broad,align=middle,frame=off]

\def\DoExample#1#2{%
  \title{#1}
  \framed{\begingroup\getbuffer[example-#2]\endgroup}
  {\tfxx\typebuffer[example-#2]}
}


\startbuffer[example-snake_lemma]%%%%%%%%%%%%%%%%%%%%%%%%%%%%%%%%%%%%%%%%%%%%%%%
\startcommutativediagram

\def\coker{{\rm coker}\,}

\lattice[golden]{
             & \obj "\ker a";   & \obj "\ker b";   & \obj "\ker c";   & \\
             & \obj A;          & \obj B;          & \obj C;          & \obj 0; \\
\obj 0,(0'); & \obj A';         & \obj B';         & \obj C';         & \\
             & \obj "\coker a"; & \obj "\coker b"; & \obj "\coker c"; & \\
};

\mor {ker a} -> {ker b} -> {ker c};
\mor A f:-> B g:-> C -> 0;
\mor[swap] 0' -> A' f':-> B' g':-> C';
\mor {coker a} -> {coker b} -> {coker c};

\mor {ker a} -> A <,a:-> A' -> {coker a};
\mor {ker b} -> B <,b:-> B' -> {coker b};
\mor {ker c} -> C <,c:-> C' -> {coker c};

\coordinate (tail) at ($(ker b)!2.4!(ker c)$);
\coordinate (head) at ($(coker b)!2.4!(coker a)$);
\coordinate (belly) at ($(tail)!0.51!(head)$);
\draw[/kD/arrows/.cd,÷>,rounded corners]
  (ker c) -- (tail) -- (tail|-belly) -- (head|-belly) -- (head) -- (coker a);

\stopcommutativediagram
\stopbuffer%%%%%%%%%%%%%%%%%%%%%%%%%%%%%%%%%%%%%%%%%%%%%%%%%%%%%%%%%%%%%%%%%%%%%

\startbuffer[example-k4_associahedron]%%%%%%%%%%%%%%%%%%%%%%%%%%%%%%%%%%%%%%%%%%
\startcommutativediagram

\def\id{{\mathbf 1}}
\catcode`ı=\active \letı\id
\catcode`×=\active \let×\otimes

\foreach [count=\n] \o in
    {((w×x)×y)×x,
     (w×(x×y))×x,
     w×((x×y)×x),
     w×(x×(y×x)),
     (w×x)×(y×x)}
  \obj at=({72*\n:9em}),(\n),"\o";

\mor 1 a_{w,x,y}×ı_z:-> 2
         a_{w,x×y,z}:-> 3
       ı_w×a_{x,y,z}:-> 4;
\mor 1   a_{w×x,y,z}:-> 5
         a_{w,x,y×z}:-> 4;

\stopcommutativediagram
\stopbuffer%%%%%%%%%%%%%%%%%%%%%%%%%%%%%%%%%%%%%%%%%%%%%%%%%%%%%%%%%%%%%%%%%%%%%

\startbuffer[example-pullback]%%%%%%%%%%%%%%%%%%%%%%%%%%%%%%%%%%%%%%%%%%%%%%%%%%
\startcommutativediagram

\lattice[comb]{
  \obj (P),A\times_ZB; & \obj B; \\
  \obj A;              & \obj Z; \\
};

\obj at=($(Z)!2!(P)$),Q;

\mor[swap] P p_1:-> A f:-> Z;
%\mor       * P_2:-> B g:-> *;
\mor[swap] Q q_1:l> A;
%\mor       * q_2:r> B;
%\mor[mid:dashed] * u:-> P;

\stopcommutativediagram
\stopbuffer%%%%%%%%%%%%%%%%%%%%%%%%%%%%%%%%%%%%%%%%%%%%%%%%%%%%%%%%%%%%%%%%%%%%%

\startbuffer[example-chain]%%%%%%%%%%%%%%%%%%%%%%%%%%%%%%%%%%%%%%%%%%%%%%%%%%%%%
\startkD[/cleanser/expand=full]

\foreach [count=\m] \a in {A,B,C}
  \foreach [count=\n] \i in {n-2,n-1,n,n+1,n+2}
    \obj at=({5em*\n,-3em*\m}),\a_{\i};

\mor C_{n+1} >-> B_{n-1};

\stopkD
\stopbuffer%%%%%%%%%%%%%%%%%%%%%%%%%%%%%%%%%%%%%%%%%%%%%%%%%%%%%%%%%%%%%%%%%%%%%



% ####################################################################### TEXT #

\starttext

\DoExample{Snake Lemma}{snake_lemma}
\DoExample{K4 associahedron}{k4_associahedron}
\DoExample{Pullback}{pullback}
\DoExample{Chain}{chain}

\stoptext

% #################################################################### THE END #
